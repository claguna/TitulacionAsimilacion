\chapter{Introducci'on}
\section{Antecedentes}

En los 'ultimos 50 a\~nos la posibilidad de observar la tierra desde el espacio
nos ha brindado la oportunidad de cambiar nuestra percepci'on sobre los sistemas 
meteorol'ogicos. Especialmente con el desarrollo de sat'elites geoestacionarios 
la capacidad de observar estos fen'omenos y su evoluci'on inici'o una nueva era en 
el trabajo cient'ifico. En un principio el simple hecho de poder observar los fen'omenos
fue un gran paso, pero la comunidad científica empez'o a trabajar en extraer informaci'on
cuantitativa de estas nuevas herramientas.

A principios de la d'ecada de 1960 los datos meteorol'ogicos, hidrol'ogicos y oceanogr'aficos
provenientes de sat'elites empezaron a tener un mayor impacto en el an'alisis del medio 
ambiente. El 7 de Diciembre de 1966, la \textit{National Aereonautics and Space Administration} (NASA)
puso en 'orbita el primer sat'elite geoestacionario \textit{Applications Technology Satellite} (ATS-1),
que ten'ia la capacidad de observar sistemas meteorol'ogicos en proceso. El sat'elite ATS-1 era
capaz de tomar una im'agen completa de la tierra cada media hora.

Poco tiempo despu'es la \textit{National Oceanic and Atmospheric Administration} (NOAA) inici'o la
operaci'on de la serie GOES con el lanzamiento de GOES-1. 


\section{Objetivo General}
Estimaci'on de precipitaci'on pluvial en el sur de M'exico empleando
t'ecnicas que se basan en datos satelitales geoestacionarios.

\section{Objetivos particulares}
\begin{itemize}
 \item Obtener la informaci'on satelital y de las estaciones pluviom'etricas para espacios temporales en donde es 
conocido que ocurrieron eventos importantes de inundaciones.
 \item Dise\~nar una base de datos geogr'afica que permita acceder a la informaci'on de manera eficiente.
 \item Desarrollar un programa en C++/Fortran que se conecte a la base de datos y procese la informaci'on para obtener
un mapa de lluvias as'i como un an'alisis de los datos obtenidos para saber que tan confiable es el algoritmo.
 \item Desarrollar una p'agina web que permita a los investigadores acceder y ejecutar el programa de manera sencilla.
\end{itemize}
