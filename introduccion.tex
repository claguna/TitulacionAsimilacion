\chapter{Introducci'on}
\section{Antecedentes}
El proyecto de 'este servicio social nace de un convenio entre el Instituto de Matem'aticas y el Instituto de Ingenier'ia.
....
- US por medio de NOAA tiene experiencia y recursos para tener informacion y sistemas
- No hay datos en M'exico
- Los datos que hay pueden no ser confiables

\section{Objetivo}
Mi participaci'on en 'este proyecto consiste en dise\~nar un sistema de informaci'on geogr'afica para poder obtener
 informaci'on sobre las precipiaciones que caen en la zona sur del pa'is.

\section{C'omputo cient'ifico}

Las Ciencias de la Computaci'on son el estudio de los fundamentos te'oricos de la informaci'on, el c'omputo y el 
desarrollo de t'ecnicas pr'acticas para la implementaci'on y aplicaci'on de las mismas en sistemas computacionales.
Las Ciencias de la Computaci'on tienen un amplio n'umero de ramas como la complejidad computacional, dise\~no de algoritmos,
graficaci'on por computadora, teor'ia de lenguajes de programaci'on, interacci'on humano-m'aquina, etc.

El C'omputo Cient'ifico, o ciencias computacionales, es una rama aplicada de las Ciencias de la Computaci'on 
que tiene el fin de estudiar y contruir modelos 
matem'aticos y t'ecnicas de an'alisis cuantitativo usando computadoras para resolver problemas cient'ificos. En forma
pr'actica, la aplicaci'on principal del c'omputo cient'ifico es la simulaci'on de fen'omenos naturales.

Algunas de las principales aplicaciones de las Ciencias Computacionales es en el An'alisis Num'erico, la F'isica y Qu'imica
Computacional y la Bioinform'atica.


En la siguiente secci'on vamos a introducir una de las aplicaciones m'as importantes de las Ciencias de la Computaci'on, las Ciencias Computacionales, la Ingenier'ia de Software, 
el Diseño de Bases de Datos y los Lenguajes de Programaci'on: 
Los Sistemas de Informaci'on Geogr'aficos.


\section{Sistemas de informaci'on geogr'aficos}
Los sitemas de informaci'on geogr'aficos (SIG) son sistemas dise\~nados para capturar, guardar, manipular, analizar,
administrar y presentar todo tipo de informaci'on geogr'afica con el fin de ser una herramienta que ayude en la toma
de desiciones.

Los SIG y el an'alisis espacial de datos se ocupan de cuantificar caracter'isticas importantes de objetos
as'i como las propiedades y atributos de las mismas. Por ejemplo, podemos caracterizar las propiedades del Monte 
Everest como su posici'on y altura.

EL dise\~no de un sistema SIG permite al usuario decidir que caracter'isticas son m'as importantes para el estudio 
de un problema. Por ejemplo, en el estudio de las zonas forestales es de importancia la precipitaci'on, el uso que 
se le da al suelo que rodea una zona forestal, la existencia de f'abricas u otros medios de contaminaci'on, etc. 
La correcta administraci'on de una zona forestal debe considerar 'estos factores y, tal v'ez con mayor importancia,
la distribuci'on espacial de los mismos.

Originalmente los SIG se localizaban principalmente en computadoras de gran capacidad y se utilizaban para mantener
registros muy especializados a los que poca gente pod'ia acceder. 
Sin embargo, con el crecimiento de Internet y el poder de c'omputo, 'estos sistemas se han puesto a la mano de 
la poblaci'on en general que requiere de medios de informaci'on confiables para realizar sus labores diarias.

Por otra parte, 'estos sistemas han tomado mayor importancia en el estudio de fen'omenos meteorol'ogicos como las
lluvias, huracanes, tormentas y otros eventos clim'aticos.